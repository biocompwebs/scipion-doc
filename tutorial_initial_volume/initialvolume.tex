%%%%%%%%%%%%%%%%%%%%%%%%%%%%%%%%%%%%%%%%%
% Design based on a template by Roberto and following the format of
% the xmipp tutorials. In turn, they seem to be based on a template
% from http://www.latextemplates.com
%%%%%%%%%%%%%%%%%%%%%%%%%%%%%%%%%%%%%%%%%

%----------------------------------------------------------------------------------------
%	PACKAGES AND OTHER DOCUMENT CONFIGURATIONS
%----------------------------------------------------------------------------------------

\documentclass[12pt]{article} % Default font size is 12pt, it can be changed here
\usepackage[english]{babel}
\usepackage[utf8]{inputenc}
\usepackage{listings} % To include source code
\usepackage{caption}
\usepackage{geometry} % Required to change the page size to A4
%\geometry{a4paper} % Set the page size to be A4 as opposed to the default US Letter
\usepackage{framed}
\usepackage{url}
\usepackage{graphicx} % Required for including pictures
\usepackage{natbib}
\usepackage{float} % Allows putting an [H] in \begin{figure} to specify the exact location of the figure
\usepackage{hyperref}
\usepackage{menukeys}

\usepackage{fancyhdr}
\pagestyle{fancy}
\fancyhf{}
\fancyhead[RO]{{Initial Volume Estimation Tutorial}} 
\fancyhead[LO]{Scipion}

%\fancyhead[RO]{{\leftmark}} 
\fancyfoot[LE,RO]{{ \thepage }}

%\usepackage{lipsum} % Used for inserting dummy 'Lorem ipsum' text into the template
\definecolor{grey}{rgb}{0.9,0.9,0.9}


\linespread{1.2} % Line spacing

%\setlength\parindent{0pt} % Uncomment to remove all indentation from paragraphs

\newcommand{\scipion}{\textsc{Scipion} }
\newenvironment{command}{\tt\begin{quote}}{\end{quote}}
\newcommand{\comm}[1]{\texttt{#1}}

\begin{document}

%----------------------------------------------------------------------------------------
%	TITLE PAGE
%----------------------------------------------------------------------------------------

\begin{titlepage}


% New command for horizontal lines. Change thickness here.
\newcommand{\HRule}{\rule{\linewidth}{0.5mm}}

\center % Center everything on the page

\includegraphics{../tutorial_common/images/scipion_logo.png}

{\large Scipion Tutorial Series}\\[1.0cm]

\textsc{\LARGE National Center for Biotechnology}\\[0.5cm]
\textsc{\Large Biocomputing Unit}\\[0.5cm]

\HRule\\[0.4cm]
{ \huge \bfseries Initial Volume Estimation Tutorial}\\[0.4cm] % Title of your document
\HRule \\[1.5cm]

%{\large \today}\\[3cm] % Date, change the \today to a set date if you want to be precise

\includegraphics[width=0.35\textwidth]{{images/00.ReconstructedVolume}.png}

\vfill % Fill the rest of the page with whitespace
%\begin{minipage}{0.4\textwidth}
\begin{flushright}
 \large
%\emph{Author:}\\
  \textsc{Scipion Team} % Your name
\end{flushright}
%\end{minipage}

\end{titlepage}


%----------------------------------------------------------------------------------------
%	OBJETIVOS
%----------------------------------------------------------------------------------------



\subsection*{Intended audience}

This tutorial is an introduction to the estimation of an initial
volume using Scipion.  Only a very limited knowledge about 3D-EM image
processing and Scipion is required, and basic computer skills.

\subsection*{We'd like to hear from you}

We have tested and verified the different steps described in this demo
to the best of our knowledge, but since our programs are in continuous
development you may find inaccuracies and errors in this text. Please,
let us know about any error you find, as well as your suggestions for
future editions by writing to
\href{mailto:scipion@cnb.csic.es}{scipion@cnb.csic.es}.

\newpage

%----------------------------------------------------------------------------------------
%	TABLE OF CONTENTS
%----------------------------------------------------------------------------------------

\tableofcontents % Include a table of contents

\newpage % Begins the essay on a new page instead of on the same page as the table of contents 


\section{General Introduction}

\subsection{Initial Volume Estimation}

Single Particle Analysis (SPA) techniques are used to obtain 3D maps
of biological complexes at near-atomic resolution. They do so by
combining tens of thousands of projection images obtained by
Transmission Electron Microscopy (TEM). In general, the reconstruction
process that leads to the final 3D map requires the use of an
approximate low resolution initial model, to be refined in further
steps.

In this tutorial we present different methods to obtain this initial
model using Scipion: Xmipp Random Conical Tilt, Xmipp 3D-RANSAC and
Eman Initial Model.

\section{Software Installation}

The first step is to download and install \scipion and related
software packages. We describe briefly the process in the next
sections. For the full documentation please refer to
\url{http://scipionwiki.cnb.csic.es/bin/view/TWiki/NewInstallation}.

\subsection{Xmipp}

One of the main components of \scipion is the Xmipp package. To get
the latest version use:

\begin{command}
git clone http://git.code.sf.net/p/newxmipp/code xmipp
\end{command}

\noindent
then change to the \directory{xmipp} directory and run the installation with

\begin{command}
./install.sh
\end{command}

\subsection{Scipion}

To get the latest version of \scipion run

\begin{command}
git clone http://git.code.sf.net/p/pyworkflow/code scipion
\end{command}

\noindent
and then change into the \directory{scipion} directory. Finally run:

\begin{command}
./scipion install --with-all-packages --with-xmipp=\$XMIPP\_HOME
\end{command}

\noindent
where \verb+$XMIPP_HOME+ points to the directory with your local
installation of Xmipp.

\subsection{Example Data}

To download the data you will work with, use the following command:

\begin{command}
scipion testdata --download rct
\end{command}

It will be downloaded in \verb+$SCIPION_HOME/data/tests/rct+.

Afterwards, you can launch the main GUI by typing:
\verb+scipion+. Create a new project by clicking on the \keys{Create
  Project} button, type a \textit{project name} and click on
\keys{OK}. This will create a new project window.

\section{Xmipp Random Conical Tilt}

One of the most widespread SPA techniques is the Random Conical Tilt (RCT) \citep{Radermacher1987a}.
In RCT the assumption is that we have a randomly distributed particle in all its conformations and positions
along a sample, so when we take a low-power image, we are getting multiple views of the same particle.
The specimen stage of the microscope is then tilted and a new low-power image is taken.

The less power is used, the less exposure of the sample. Having a low
exposure is a cryoEM constraint on biological objects.  As a result,
we get images with low signal-to-noise ratio (SNR). With these two
images, every processing program will need to relate each particle to
its tilted version in order to get different views of the specimen and
estimate the tilt angle.

\begin{figure}
\centering
\includegraphics[width=0.75\textwidth]
{{images/01.RCTprojection}.png}
\caption{How the projections vary with the tilt.}
\label{RCTprojection}
\end{figure}

\subsection{Import Tilted Pairs Micrographs}

Go to the \keys{View} selection menu on the top left panel and select
\textbf{Random Conical Tilt}.

The first step is to import the tilted pairs micrographs to your scipion project. To do this,
select \menu{Import Micrographs pairs} protocol. In \textit{Pattern untilted} and \textit{Pattern tilted}  you must indicate where your untilted 
and tilted micrograph files are stored (clicking on \keys{browse} button). 
The complete patterns are:

\begin{verbatim}
$SCIPION_HOME/data/tests/rct/micrographs/F_rct_u_*.tif
$SCIPION_HOME/data/tests/rct/micrographs/F_rct_t_*.tif
\end{verbatim}

\newpage

In this first version pairs assignment is done by micrograph order but in next versions a wizard 
will be provided.

Modify the parameters of the Import Micrographs protocol according to the ones
shown in figure~\ref{ImportMics}.

\begin{figure}
\centering
\includegraphics[width=0.75\textwidth]
{{images/02.ImportMics}.png}
\caption{Import Micrographs protocol GUI.}
\label{ImportMics}
\end{figure}

When you have completed the form, click on the \keys{Execute} button. 

Once the protocol has finished you can press on the \keys{Analyze results} button and a new pop-up GUI will open
where you can see the imported micrographs pairs as shown in figure ~\ref{ImportResults}.

\begin{figure}
\centering
\includegraphics[width=0.75\textwidth]
{{images/03.ImportResults}.png}
\caption{Import Micrographs results.}
\label{ImportResults}
\end{figure}

\subsection{Particle Picking}

After micrographs have been imported you need to pick tilted and untilted particle pairs.
Go to the \menu{Picking micrographs pairs} protocol  and a form as shown in figure~\ref{PickingProtocol} 
will appear. Select the Micrographs Tilt Pair object produced by the import protocol as input and 
click on the \keys{Execute} button. The Xmipp particle picking GUI will open.

If you find some particles difficult to see due to noise and contrast of the images, you can use the 
different filters offered by the Xmipp particle picking GUI on the \textbf{Filters} menu.
For instance you might select the \menu{Filters > Brightness/Contrast} filter and press the \keys{Auto} button that will 
set the brightness and contrast to the optimal values for the selected micrograh. Then change to other micrographs 
and do the same. The resulting interface will be much easier to start picking.

\newpage

\begin{figure}
\centering
\includegraphics[width=0.75\textwidth]{{images/04.PickingProtocol}.png}
\caption{Picking pairs protocol}
\label{PickingProtocol}
\end{figure}

The procedure to do the picking is:

1.  Find the first particle in both untilted and tilted. First pick the untilted and then its tilted
pair.

2.  Pick three more particles like the first one, having special care in correcting
the tilted one, so the tilt estimation can be properly set. During the first particles picking,
the manual error correction will help the algorithm to learn and propose a better tilt adjust-
ment. For detecting the error, it is better to zoom in the micrograph. Note that zooming in
or out in the untilted micrograph will replicate in the tilted one, but not in the other side.
To zoom in, press \keys{shift} and scroll up, to zoom out, press \keys{shift} and scroll down.

3.  Finally continue picking just the untilted (tilted will automatically be picked) but keeping
an eye on the tilted, in order to prevent from deviations due to accumulated error. The amount 
of particles to pick for a good reconstruction varies depending on the tilt angle,
the quality of the micrographs and many other factors, but a good approach to start could
be 2000 particles.

However if you don't feel like picking manually thousands of particles you can import the already picked 
particles by clicking on \menu{File > Import coordinates} as shown on figure~\ref{PickingGUI} 
specifying the following path:

\begin{verbatim}
$SCIPION_HOME/data/tests/rct/positions
\end{verbatim}

\begin{figure}
\centering
\includegraphics[width=0.75\textwidth]{{images/05.PickingGUI}.png}
\caption{Xmipp particle picking pairs GUI}
\label{PickingGUI}
\end{figure}

Take into account that the protocol status will not changed to \textbf{Finished} but will remain as \textbf{Interactive}.
This allows you to execute it again if you wish to pick more particles.

\subsection{Extract Particle Pairs}

Now that coordinates pairs are picked they have to be extracted as particle pairs.
To do so select the \menu{Extract particle pairs} protocol and fill in the parameters as shown in figure~\ref{ExtractPairs}.
Although particles were picked with a size of 120 pixels the extract will downsample them by a factor of two to speed up next steps.

Once the protocol has finished you can review the particles by clicking on the \keys{Analyze results} button as seen in figure~\ref{ExtractResults}.
%If you don't like some of the particles you can disable them and create a subset with the good ones. Just click on the \textbf{TiltPairs} red button and give the subset a name.

\begin{figure}
\centering
\includegraphics[width=0.75\textwidth]{{images/06.ExtractPairsProtocol}.png}
\caption{Extract particle pairs protocol}
\label{ExtractPairs}
\end{figure}

\begin{figure}
\centering
\includegraphics[width=0.75\textwidth]{{images/07.ExtractPairsResults}.png}
\caption{Extract particle pairs results}
\label{ExtractResults}
\end{figure}

\subsection{Classification with Xmipp CL2D}
Once the particles are extracted, they have to be classified, in order to group those belonging to different state of the particle 
or even those which represent an error in the picking. 

For this tutorial you will perform a CL2D classification from Xmipp package but you could choose any other 2D classification protocol, 
which can be found under the \textbf{Protocols SPA} view grouped as \menu{2D > Classify}.

Select the set of untilted particles produced by the previous step,  %or the ones on the subset if you chose to remove some of the particles.
fill in the rest of the parameters as shown in figure~\ref{CL2D} and press \keys{Execute}.

This protocol will take some time to finish (depending on your computer power and number of \textbf{mpis} choosen). 
Once it has finished you can visualize the classes obtained by clicking on \keys{Analyze results} button.

This classes show mean images that represent your particles, but to be sure that they are representing with fidelity your particle, 
at least some basic assessment steps are needed. 

Yo should be using the stable core classes. These are composed by those particles that have stayed through all iterations together. 

\begin{figure}
\centering
\includegraphics[width=0.65\textwidth]{{images/08.CL2Dprotocol}.png}
\caption{CL2D protocol}
\label{CL2D}
\end{figure}

Each class should contain between 100 and 200 particles more or less. Less than 100 will result in reconstructions of low quality, 
so the estimation is not being properly done. Also, more than 200-250 particles would result in a wrong estimation because particles 
very different could be mixing in the same class. 

Apart from that, there is also a subjective assessment, and it comes from what the user is looking for. 
Each class shows you a mean image that can give you an idea of what you will find if you start a reconstruction with those images. 
The user must select those volumes that are more promising. In the case shown on our figures, classes 3 and 7 have been discarded, 
that seem to show particles in a top view (or just one part of the particle).

Then, as done with particles extracted, you should create a subset of classes to be used as input for next step, 
as shown in figure~\ref{CL2Dresults}


\begin{figure}
\centering
\includegraphics[width=0.65\textwidth]{{images/09.GoodClasses}.png}
\caption{CL2D results}
\label{CL2Dresults}
\end{figure}

\subsection{Initial Volume Generation with Random Conical Tilt}

Now that you have the desired classes, only the RCT reconstruction step is left. 

Select protocol \textbf{Random Conical Tilt} and fill in the form as shown in figure~\ref{RCTprotocol}.

You have to provide the particles tilt pair object produced by the extraction (or a subset) and the particles or classes 
you want to reconstruct. Take into account that if you choose particles they should contain alignment information.
The \textbf{Thin Object} option must be set to “Yes” if the particle is not more or less spherical. 
When the object is going to change its width significantly between the untilted and the tilted image, we can considered this a “thin object”. 
Reconstruction can also low-pass filter the volumes after the process is done, producing both filtered and not filtered volumes as output.

In order to visualize the volumes click on \keys{Analyze results} after the protocol has finished, and a window as the one on figure~\ref{RCTresults} 
will show up.

\begin{figure}[H]
\centering
\includegraphics[width=0.6\textwidth]{{images/10.RCTprotocol}.png}
\caption{Xmipp RCT protocol}
\label{RCTprotocol}
\end{figure}

\begin{figure}
\centering
\includegraphics[width=0.6\textwidth]{{images/11.RCTresults}.png}
\caption{Xmipp RCT output: Filtered volume.}
\label{RCTresults}
\end{figure}

\section{3D Ransac}

In this section, we present how to use Xmipp 3D-RANSAC approach, which can obtain a reliable low resolution initial volume 
from sets of macromolecular projection images without a priori information ( Vargas et al. (2014)). 

\subsection{Initial Volume Generation with 3D Ransac}

Change to the \textbf{Protocols SPA} view if you are not already there, go to \menu{3D > Initial Volume > xmipp - ransac} protocol. 

As input we will use the classes obtained on Random Conical Tilt section so we can compare the generated volumes using different 
techniques. Also, fill in the other form parameters as shown in figure~\ref{RansacProtocol} and explained below. 

\begin{figure}
\centering
\includegraphics[width=0.75\textwidth]{{images/15.RansacProtocol}.png}
\caption{Xmipp Ransac protocol.}
\label{RansacProtocol}
\end{figure}

There is not symmetry so leave c1 and the angular sampling, in degrees, defines how fine the projection gallery from the volume is explored. 
The number of RANSAC iterations are the number of random volumes which are generated, in our case 100. 
We select performing dimensionality reduction, with a number of grids per dimension equal to 6, and therefore each
random volume is generated from 36 (6x6) classes. 
The inlier threshold is a correlation based cutoff to determine if a class is an inlier of the generated random volume or not. 
Note that in cases where the classes have very low noise this value must be increased ( 0.8-0.95) and, on the other hand, 
in cases with high noise, this value must be decreased. 
The \textbf{Number of best volumes to refine} are the number of volumes, among all the generated ones (in this case 100), that
are going to be further refined using a projection matching approach by the selected \textbf{Number of iterations to perform to refine the volumes}. 
Finally, \textbf{Max frequency of the initial volume} is the maximum frequency of the initial volume in Angstroms. 
Press the \keys{Execute} button and wait until the protocol has finished when output volumes could be visualize by clicking 
on \keys{Analyze Results} button as seen in figure~\ref{RansacResults}

\begin{figure}
\centering
\includegraphics[width=0.75\textwidth]{{images/16.RansacResults}.png}
\caption{Xmipp Ransac output volumes.}
\label{RansacResults}
\end{figure}

\newpage

\section{Eman initial model}

In this section we present how to obtain an initial module in Scipion using Eman initial model.

\subsection{Initial Volume Generation with Eman}

Change to the \textbf{Protocols SPA} view if you are not already
there, go to \menu{3D > Initial Volume > eman2 - initial model}
protocol.  As we did before use the classes obtained on Random Conical
Tilted section as input so we can compare the generated volumes using
different techniques. Also, fill in the other form parameters as shown
in figure~\ref{EmanProtocol}

Once the protocol has finished go and view the generated output volumes by clicking on \keys{Analyze results}, as you can see in figure~\ref{EmanResults}.

\begin{figure}[H]
\centering
\includegraphics[width=0.75\textwidth]{{images/17.EmanProtocol}.png}
\caption{Eman protocol.}
\label{EmanProtocol}
\end{figure}

\begin{figure}
\centering
\includegraphics[width=0.75\textwidth]{{images/18.EmanResults}.png}
\caption{Eman generated output volumes.}
\label{EmanResults}
\end{figure}

\bibliographystyle{apalike}
\bibliography{../tutorial_common/em.bib}

\end{document}
